% Please do not change the document class
\documentclass{scrartcl}

% Please do not change these packages
\usepackage[hidelinks]{hyperref}
\usepackage[none]{hyphenat}
\usepackage{setspace}
\doublespace

% You may add additional packages here
\usepackage{amsmath}

% Please include a clear, concise, and descriptive title
\title{Does playing First Person Shooter games Cause Aggressive and Violent Thoughts in Users? }

% Please do not change the subtitle
\subtitle{Comp 230 - Ethics and Professionalism}

% Please put your student number in the author field
\author{1605629}

\begin{document}

\maketitle

\abstract{ In recent times, much of the media has portrayed games in a negative way. One such way is that they cause violent and aggressiver thoughts in their players. This is then used as a suggested cause for murder crimes. The first person shooter genre is the biggest contributor to this negative reputation for inciting violence. This kind of reputation can be very negative for the games industry as a whole and so this paper will go into whether first person shooter games do actually cause violent thoughts in the people who play them and how much games companies are responsible for ensuring they don't. }

\section{Introduction}
In the modern media, games have begun to get a reputation of causing violent and aggressive thoughts and tendencies in people who have played them. Although not all genres have this reputation, it could be argued that first person shooters (FPS) are the genre of games that have the worst reputation for causing violent tendencies. However, at a first glance much of the accusations appear to be very opinionated and lacking in factual evidence. This paper will look into research in the area and  will aim to distinguish whether FPS games do cause aggressive and violent thoughts and tendencies in their players.

\section{Arousal}
Although it has been found that the inclusion of blood to violent games increases user arousal \cite {ballard1996mortal}, this does not necessarily mean that it increases levels of aggression of its users. This is because arousal is actually the level of excitement of the player. A high level of arousal can cause a player to pay more attention to and concentrate on the game a lot more than a player with low levels of arousal would \cite {jeong2015brand}. If a high level of arousal is attained while playing FPS games it could cause a player's aggression levels to increase \cite{jeong2015brand}, however, this would, in most cases, only be for the duration of the game, as arousal levels would drop quickly after the user stops playing. It has further been observed during a research project using Serious Sam as an example that the addition of blood and gore into a violent first person game often has no effect on stimulation levels \cite{ashbarry2016blood}. 

\section{Toxic Inhibition}
An interesting phenomenon that is often exhibited by users that play violent FPS games is Toxic Inhibition. This can be defined as the tendency for people to use unnecessarily harsh or rude language, criticism, anger or even threats while online that does not match their real world personality \cite{rawn2008examining}. Although it has been found to be an online phenomenon, there hasn't been any meaningful research into how long this effect last after playing games \cite{rawn2008examining}. Riot Games have spent a lot of time trying to reduce the toxicity of their community and as such, shows that games companies are to some extent responsible for the limitation of toxic inhibition.

\section{Physical Stimuli - The Priming Effect}
Another factor that could increase levels of aggression is a physical stimulus such as a realistic controller. The idea behind this is that the realistic controller would increase the users perceived presence within the game, making it a much more personal affair. This would then increase the players level of aggression as they would feel like the actions happening in game were happening to them in the real world. It has been proven many times over that the addition of a weapon stimuli has a priming effect on the accessibility of aggression related thoughts. \cite{kim2011effects}. Kim et al noted during their research that adding in a realistic controller did increase the users perceived presence and thus also their physical aggression levels when compared to the use of a mouse and keyboard \cite {kim2011effects}. One way companies reduce this is by making their gun controllers look unrealistic.

\section {Violence desensitisation / Predisposition}
A further factor behind this could be whether the users are already desensitised towards violence or have a predisposition to violence. It has been widely studied and mutually concluded that exposure to violence is more likely to lead to violent attitudes later in life \cite{kim2011effects}. The exposure to violence in both real life and entertainment can result in reduced empathy \cite{kim2011effects}. It has been noted in a study by Kim et al that people exposed to real-life violence showed a significantly lower state of aggression after the game than those without any past experience \cite{kim2011effects}. This shows that people with past experience of violence show a greater level of acceptance and tolerance for gun related aggressive behaviours \cite{kim2011effects}. This proves to be a worrying prospect as the availability of firearms in countries such as the United States of America is high. This phenomenon implies that violent FPS games could act as a catalyst for violent crimes committed by users with past history of exposure to violence.

\section{Violence and Aggression}
The idea that games increase violence in their players is a hotly debated topic with many researchers coming to polar opposite decisions \cite {zhang2009violent}. It has been found that increased sense of presence in a game increases verbal hostility and violence while playing violent FPS games \cite{nowak2008causes}. . However, it has also been found that peoples personality affects their aggression within games just as it would in real life decisions \cite{murzyn2016our}. This is interesting, as it means that if an FPS game gives a player options to progress the storyline in different ways, both violent and non-violent, people with higher violence levels before playing will choose the more violent route and people with lower violence levels will naturally choose a nonviolent route \cite{murzyn2016our}. In addition, other studies have also found that ,for users with minimal exposure to violence in early life, playing violent FPS games has a negligable effect on the level of aggression \cite{fumhe2015violent}. An interesting study found that in a study of 710 people playing a GTA like FPS game, as the graphical fidelity of the game increased the users violence levels decreased \cite{zendle2015higher}. This means that as a games graphics improve users exhibit a lower level of aggression. Przybylski et al also found in a study that games which have incompetence inducing controls tend to increase a user's aggression as the controls go against a persons basic need for competency \cite{przybylski2014competence}.


\section{Conclusion}
Although the media often places the blame of violent crimes on violent games, there is little evidence to back this up. In fact there is a growing pool of evidence showing that games can at most be a catalyst for violent acts, albeit still at a lower level than other forms of media such as films and television series \cite{rawn2008examining}. The majority of evidence points towards the fact that games do increase violent and aggressive tendencies in the user while playing, due to Toxic inhibition, with varying levels of extremity based on the users previous exposure to violence. However, these effects are not known to last past the playing experience. It has even been shown that games can act as an effective release of anger and therefore act as a therapy tool, reducing the uses over all aggression level \cite {lee2016role}. This evidence shows that most FPS games do not increase violent tendencies in their users.


\bibliography{references}
\bibliographystyle{ieeetran}


\end{document}
